\chapter{代数理论}

目前代数理论可以用于解决组合逻辑与推广递次的电路压缩问题。本章内容面向有线性代数基础的读者。没有线性代数基础的读者需要预先学习向量、矩阵、线性方程组、行列式、秩的运算。群论基础对于本章内容的理解有帮助但是没有必要。

\section{域$\mathbb{Z}_2$简介}
把整数分成奇数和偶数,则有如下运算规律:
\begin{center}
\begin{tabular}{|c|cc|}
	\hline
	$+$&偶&奇\\\hline
	偶&偶&奇\\
	奇&奇&偶\\\hline
\end{tabular}
\begin{tabular}{|c|cc|}
	\hline
	$\times$&偶&奇\\\hline
	偶&偶&偶\\
	奇&偶&奇\\\hline
\end{tabular}
\end{center}

$\mathbb{Z}_2=\{0,1\}$表示整数除以2所得余数的集合,则0代表偶数,1代表奇数,0和1之间的加法和乘法运算遵守上述奇偶的运算规律,即
\begin{center}
\begin{tabular}{|c|cc|}
	\hline
	$+$&0&1\\\hline
	0&0&1\\
	1&1&0\\\hline
\end{tabular}
\begin{tabular}{|c|cc|}
	\hline
	$\times$&0&1\\\hline
	0&0&0\\
	1&0&1\\\hline
\end{tabular}
\end{center}

另外,定义除法为乘法的逆运算,即$0\div 1=0$、$1\div 1=1$,除数不能为0。定义减法为加法的逆运算,由于$\mathbb{Z}_2$中恰好有$1=-1$,减法和加法的运算规则完全一样。

加法和乘法满足交换律和结合律,乘法对加法满足分配律。

$\mathbb{Z}_2$上的乘法和加法可以很好地描述泰拉瑞亚电路逻辑,如\autoref{fig83}所示。把多根线接到同一个输出上,相当于做加法;与逻辑相当于多个输入做乘法。

\begin{figure}[!htp]
	\centering
	\subfloat[]{
		\label{i350:351:2}
		\adjincludegraphics{images/350.png}%
		\quad%
		\adjincludegraphics{images/351.png}
	}%
	\subfloat[]{
		\label{i352:353:2}
		\adjincludegraphics{images/352.png}%
		\quad%
		\adjincludegraphics{images/353.png}
	}%
	\caption{
		\protect\subref{i350:351:2}加法:$B=A+C$;
		\protect\subref{i352:353:2}乘法:$C=AB$。
	}\label{fig83}
\end{figure}

\section{$\mathbb{Z}_2$上的线性代数}
通过定义$\mathbb{Z}_2$上的加减乘除就可以导出$\mathbb{Z}_2$上的线性代数,基本理论也和实/复数域上线性代数没有什么区别。唯一需要注意的是,$\mathbb{Z}_2$上的多项式环结构不同于实/复数域,所以相应的特征值理论有区别,这个区别将在后续小节中细说。本小节将侧重于解释为什么我们需要用到$\mathbb{Z}_2$上的线性代数。

$\mathbb{Z}_2$上的向量和矩阵在电路中有实际意义,因为接线的本质就是矩阵。

%\begin{example}{}{}
%	一个逻辑门上的普通逻辑灯状态可以排列成一个向量,亮对应$\mathbb{Z}_2$中的1,灭对应$\mathbb{Z}_2$中的0。
%\end{example}
\begin{example}{}{}
	一个七段线显示器的状态可以表示成一个$\mathbb{Z}_2$上的7维向量。考虑如图所示的七段线显示结构,七个输入并不直接对应七段线。输入的状态表示为$\mathbf{x}=(x_1,\dots,x_7)$,七段线状态表示为$\mathbf{y}=(y_1,\dots,y_7)$。
	
	\begin{center}
	\begin{tikzpicture}[scale=0.5]
		\draw[rounded corners, fill] (0,0) rectangle node[white] {$y_7$} (3,1);
		\draw[rounded corners, fill] (0,1) rectangle node[white] {$y_5$} (-1,4);
		\draw[rounded corners, fill] (3,1) rectangle node[white] {$y_6$} (4,4);
		\draw[rounded corners, fill] (0,4) rectangle node[white] {$y_4$} (3,5);
		\draw[rounded corners, fill] (0,5) rectangle node[white] {$y_2$} (-1,8);
		\draw[rounded corners, fill] (3,5) rectangle node[white] {$y_3$} (4,8);
		\draw[rounded corners, fill] (0,8) rectangle node[white] {$y_1$} (3,9);

		\draw[red, ultra thick] (-0.25,3.75) -- (-0.25,0.75) -- (3.25,0.75) -- (3.25,3.75) -- (10,3.75) node[thmcoltext, anchor=west] {$x_5$};
		\draw[cyan, ultra thick] (0.25,0.5) -- (3.5,0.5) -- (3.5,3.5) -- (3.5,3.0) -- (10,3.0) node[thmcoltext, anchor=west] {$x_6$};
		\draw[green, ultra thick] (3.75,1.25) -- (3.75,3.5) -- (3.75,2.25) -- (10,2.25) node[thmcoltext, anchor=west] {$x_7$};

		\draw[red, ultra thick] (-0.25,5.25) -- (-0.25,8.25) -- (3.25,8.25) -- (3.25,5.25) -- (10,5.25) node[thmcoltext, anchor=west] {$x_3$};
		\draw[cyan, ultra thick] (0.25,8.5) -- (3.5,8.5) -- (3.5,5.5) -- (3.5,6) -- (10,6) node[thmcoltext, anchor=west] {$x_2$};
		\draw[green, ultra thick] (3.75,7.75) -- (3.75,5.5) -- (3.75,6.75) -- (10,6.75) node[thmcoltext, anchor=west] {$x_1$};

		\draw[brown, ultra thick] (-0.5,1.25) -- (-0.5,7.75);
		\draw[brown, ultra thick] (-0.5,4.5) -- (10,4.5) node[thmcoltext, anchor=west] {$x_4$};
	\end{tikzpicture}
	\end{center}
	
	七个输入和七段线的对应关系可以用一个矩阵表示:
	
	\[\begin{array}{c|ccccccc}
			& y_1 & y_2 & y_3 & y_4 & y_5 & y_6 & y_7 \\\hline
		x_1 &     &     &  1  &     &     &     &     \\
		x_2 &  1  &     &  1  &     &     &     &     \\
		x_3 &  1  &  1  &  1  &     &     &     &     \\
		x_4 &     &  1  &     &  1  &  1  &     &     \\
		x_5 &     &     &     &     &  1  &  1  &  1  \\
		x_6 &     &     &     &     &     &  1  &  1  \\
		x_7 &     &     &     &     &     &  1  &     \\\hline
	\end{array}\]
	
	%表格中每行表示一个输入覆盖的段,可以理解为一个7维行向量;每列表示经过一个分段的输入,可以理解为一个7维列向量。去掉表头,这个表格就是一个$\mathbb{Z}_2$上的$7\times 7$\myind{矩阵},记为$\mathbf{A}$。
	
	%一个分段的值等于经过它的输入的和,例如$y_3=x_1+x_2+x_3$。这可以写成$y_3=1x_1+1x_2+1x_3+0x_4+0x_5+0x_6+0x_7$,即$y_3$是输入$\mathbf{x}$和矩阵中对应列向量的\myind{向量积}:
	
	%\[
	%	y_3=\mathbf{x}(1,1,1,0,0,0,0)^\top,
	%\]
	记这个矩阵为$\mathbf{A}^\top$,七段线的初始状态为$\mathbf{b}$,则分段与输入的关系可以表示为$\mathbf{y}=\mathbf{A}\mathbf{x}+\mathbf{b}$.
	
\end{example}

一般的组合逻辑可以看作集合。
\begin{definition}{}{}
	$n$输入的组合逻辑和$\mathbb{Z}_2^n$的子集存在一一对应关系。设某逻辑为$n$元函数$f$,则相应的集合为$f^{-1}(1)=\{\mathbf{v}\in\mathbb{Z}_2^n:f(\mathbf{v})=1\}$,这个集合称为$f$的\emph{特征集}。一个组合逻辑的输入数称为该逻辑的\emph{维数},特征集的秩称为该逻辑的\emph{秩}。
\end{definition}
\begin{example}{}{}
	$n$维异或逻辑的特征集是$\mathbb{Z}_2^n$的标准基。
\end{example}
\begin{theorem}{TNoName \& putianyi888}{}
	假设某$n$维组合逻辑可以用单异或门实现,则该异或门的最优灯数不超过$n+1$。
\end{theorem}
\begin{proof}
	用$V=\mathbb{Z}_2^n$表示输入空间,$U=\mathbb{Z}_2^m$表示灯的状态空间。用矩阵$\boldsymbol{L}\in\mathbb{Z}_2^{m\times n}$表示接线,向量$\boldsymbol{b}\in U$表示灯的初始状态,则$L:V\to U, \boldsymbol{v}\mapsto \boldsymbol{L}\boldsymbol{v}+\boldsymbol{b}$表示从输入到灯的状态的映射。设某$n$维逻辑为$f:V\to\mathbb{Z}_2$,实现该逻辑的异或门为$g:U\to \mathbb{Z}_2^n$。假设$m>n+1$。
	
	先考虑齐次情况$\boldsymbol{b}=\boldsymbol{0}$。由于$\dim(V)=n$,$L(V)$是$U$的一个$n'\le n$维线性子空间。因为$m>n+1\ge n'+1$,取$U/L(V)$的一组基$\boldsymbol{A}=(\boldsymbol{\alpha}_1,\dots,\boldsymbol{\alpha}_{m-n'})$。由于$g$是异或逻辑,$g^{-1}(1)$是$U$的标准基,且$L(f^{-1}(1))=g^{-1}(1)\cap L(V)$。记$L(f^{-1}(1))$为$\boldsymbol{B}=(\boldsymbol{\beta}_1,\dots,\boldsymbol{\beta}_r)$,取$L(V)/\spanspace(L(f^{-1}))$的一组基$\boldsymbol{C}=(\boldsymbol{\gamma}_1,\dots,\boldsymbol{\gamma}_{n'-r})$。显然$(\boldsymbol{B}|\boldsymbol{C}|\boldsymbol{A})$是$U$的一组基。
	
	构造$U$的另一组基$\boldsymbol{F}=(\boldsymbol{B}|\boldsymbol{C}+\boldsymbol{\alpha}_1|\boldsymbol{A})$,其中$\boldsymbol{C}+\boldsymbol{\alpha}_1$表示将$\boldsymbol{\alpha}_1$加到$\boldsymbol{C}$的每一列上。记$F^{-1}:\boldsymbol{v}\mapsto\boldsymbol{F}^{-1}\boldsymbol{v}$是将$\boldsymbol{F}$映到标准基的线性映射。考虑复合映射$F^{-1}L$,有
	\begin{enumerate}
		\item 当$\boldsymbol{v}\in f^{-1}(1)$时,$L(\boldsymbol{v})\in\boldsymbol{B}$,从而$F^{-1}L(\boldsymbol{v})$是一个标准基向量。
		\item 当$\boldsymbol{v}\in V\backslash f^{-1}(1)$时,$L(\boldsymbol{v})\notin\boldsymbol{F}$,从而$F^{-1}L(\boldsymbol{v})$不是一个标准基向量。
		\item 当$\boldsymbol{v}\in V$时,$L(\boldsymbol{v})\in\spanspace(\boldsymbol{B}|\boldsymbol{C})\subseteq\spanspace(\boldsymbol{B}|\boldsymbol{C}+\boldsymbol{\alpha}_1|\boldsymbol{\alpha}_1)$,从而$F^{-1}L(\boldsymbol{v})$的后$(m-n'-1)$个分量为0。
	\end{enumerate}
	综上三条性质,接线$F^{-1}L$与$L$对于异或门是等价逻辑,且在接线$F^{-1}L$的情况下,异或门的后$(m-n'-1)$个灯恒灭,所以这些灯可以去掉,剩下$(n'+1)$个灯。
	
	对于非齐次情况$\boldsymbol{b}\ne\boldsymbol{0}$,如果$\boldsymbol{b}\in \boldsymbol{L}V$,那么$L(V)=\boldsymbol{L}V$,退化为齐次情况。如果$L(V)\cap g^{-1}(1)=\emptyset$,则$f(V)=\{0\}$,退化为平凡逻辑。
	
	一般地,取$\boldsymbol{b}'\in L(V)\cap g^{-1}(1)$,那么$L(V)=\boldsymbol{L}V+\boldsymbol{b}'$。
\end{proof}

\section{$\mathbb{Z}_2$上的多项式}


% \section{向量与矩阵}
% $\mathbb{Z}_2$上的向量即为一组有序的$\mathbb{Z}_2$元素。向量组成向量空间。





% \begin{definition}{}{}
% $\mathbb{Z}_2^n=\{(x_1,\dots,x_n): x_k\in\mathbb{Z}_2\}$是$\mathbb{Z}_2$上的$n$维向量空间。$\mathbb{Z}_2^n$中的元素是$\mathbb{Z}_2$上的$n$维向量。
% \end{definition}

% 向量的加减法定义为对应元素加减:
% \[
	% (x_1,\dots,x_n) + (y_1,\dots,y_n) = (x_1+y_1,\dots,x_n+y_n).
% \]

% 数字乘向量定义为用数字乘向量中的每个元素:
% \[
	% \lambda(x_1,\dots,x_n) = (\lambda x_1,\dots,\lambda x_n).
% \]
% 由于$\mathbb{Z}_2$中仅有0和1两个数字,所以向量的数乘可以列举为
% \[
	% 1\mathbf{x}=\mathbf{x},\qquad 0\mathbf{x}=\mathbf{0},
% \]
% 其中粗体小写字母表示向量,$\mathbf{0}$表示零向量,即向量中所有元素都是0。



% $\mathbb{Z}_2$元素也可以有序地组成矩阵。矩阵组成矩阵空间。
% \begin{definition}
% $\mathbb{Z}_2^{m\times n}=\left\{\begin{pmatrix}
	% x_{11}&x_{12}&\cdots&x_{1n}\\
	% x_{21}&x_{22}&\cdots&x_{2n}\\
	% \vdots&\vdots&\ddots&\vdots\\
	% x_{m1}&x_{m2}&\cdots&x_{mn}
% \end{pmatrix}: x_{jk}\in\mathbb{Z}_2\right\}$是$\mathbb{Z}_2$上的$m\times n$维矩阵空间。$\mathbb{Z}_2^{m\times n}$中的元素是$\mathbb{Z}_2$上的$m\times n$维矩阵。
% \end{definition}

% 矩阵也可以看作向量的向量。一个$m\times n$维矩阵既可以理解为由$m$个$n$维行向量组成的列向量也可以理解为由$n$个$m$维列向量组成的行向量。$n$维行向量也可以看作$1\times n$维矩阵,$m$维列向量也可以看作$m\times 1$维矩阵。

% 按照向量加减法与数乘的定义,矩阵加减法也定义成对应元素加减,矩阵数乘也定义成用数字乘矩阵中的每个元素。

% 矩阵一般用粗体大写字母表示。零矩阵用$\mathbf{O}$表示。